%%%%%%%%%%%%%%%%%%%%%%%%%%%%%%%%%%%%%%%%%%%%%%%%%%%%%%%%%%%%%%%%%%%%%%%%%%%%%%%%
%%%%%%%%%%%%%%%%%%%%%%%%%%%%%%%%%%%%%%%%%%%%%%%%%%%%%%%%%%%%%%%%%%%%%%%%%%%%%%%%
% PACKAGES
%%%%%%%%%%%%%%%%%%%%%%%%%%%%%%%%%%%%%%%%%%%%%%%%%%%%%%%%%%%%%%%%%%%%%%%%%%%%%%%%
%%%%%%%%%%%%%%%%%%%%%%%%%%%%%%%%%%%%%%%%%%%%%%%%%%%%%%%%%%%%%%%%%%%%%%%%%%%%%%%%
% Algorithms
\usepackage[linesnumbered,vlined,ruled,commentsnumbered]{algorithm2e}

% Babel
\usepackage[english]{babel}

% Code writing
	%\usepackage[procnames]{listings}

% Font
\usepackage[utf8]{inputenc}
\usepackage[T1]{fontenc}
\usepackage[fleqn]{amsmath}
\usepackage{amssymb,amsthm,amsfonts}
\usepackage{mathtools}
\usepackage{eucal}
\usepackage{textcomp}
\usepackage{dsfont}
\usepackage{bbm}
\usepackage{inconsolata}

% Footnote


% Hyperref
\usepackage[hyphens]{url}
\usepackage{cite}
\usepackage{hyperref}
\usepackage{nameref}

% Images
\usepackage[pdftex]{graphicx}
%\usepackage{subfigure}
\usepackage{subfig}
\usepackage{eso-pic}
\usepackage{caption}
\usepackage{wrapfig}
\usepackage{float}

% List
\usepackage{enumerate}

% SI units
\usepackage{siunitx}

% Standalone
\usepackage[subpreambles=true]{standalone}
\usepackage{import}

% Tables
\usepackage{tabularx}
\usepackage{booktabs}
\usepackage{multirow}

% TiKz and graphs
\usepackage{pgf,tikz,pgfplots}
% \usepackage{gnuplottex}
\usepackage{bm}
\usepackage{relsize}
\usepackage{tikz-3dplot}
\usepackage[compat=1.1.0]{tikz-feynman}
\usepackage{circuitikz}
\pgfplotsset{compat=1.16}

% Typeset
\usepackage[top=2cm,bottom=2cm,left=2cm,right=2cm]{geometry}
\usepackage{fancyhdr}
\usepackage{indentfirst}
\usepackage{titlesec}
\usepackage{setspace}
\usepackage{xspace}
% \usepackage{parskip}  % Elimina il separatore a inizio paragrafo
\usepackage{afterpage}
\usepackage{comment}

%Python
\usepackage{xcolor}
\usepackage{listings}
\usepackage{framed}

%Per scrivere matrice identità
%Per semplificazione formule
\usepackage{cancel}

% Others
\usepackage{empheq}
\usepackage{soul}
%Evidenziare testo con mdframed
\usepackage{mdframed}
%Note a margine
\usepackage{marginnote}
\usepackage{mparhack}
%Display data
\usepackage{datetime}
%Physics
\usepackage{physics}

\usepackage[many,most,theorems]{tcolorbox}





%%%%%%%%%%%%%%%%%%%%%%%%%%%%%%%%%%%%%%%%%%%%%%%%%%%%%%%%%%%%%%%%%%%%%%%%%%%%%%%%
%%%%%%%%%%%%%%%%%%%%%%%%%%%%%%%%%%%%%%%%%%%%%%%%%%%%%%%%%%%%%%%%%%%%%%%%%%%%%%%%
% NEW COMMANDS
%%%%%%%%%%%%%%%%%%%%%%%%%%%%%%%%%%%%%%%%%%%%%%%%%%%%%%%%%%%%%%%%%%%%%%%%%%%%%%%%
%%%%%%%%%%%%%%%%%%%%%%%%%%%%%%%%%%%%%%%%%%%%%%%%%%%%%%%%%%%%%%%%%%%%%%%%%%%%%%%%
%% User defined commands
\newcommand{\mathcolorbox}[2]{\colorbox{#1}{$\displaystyle #2$}}
\newcommand{\hlfancy}[2]{\sethlcolor{#1}\hl{#2}}

%% User defined math commands
\newcommand{\id}{\mathbbm{1}}
\newcommand{\floor}[1]{\left \lfloor #1 \right \rfloor}
\newcommand{\ceil}[1]{\left \lceil #1 \right \rceil}
\renewcommand{\Re}[1]{\operatorname{\mathbb{R}e}\left[ #1 \right]}
\renewcommand{\Im}[1]{\operatorname{\mathbb{I}m}\left[ #1 \right]}

\newtcbtheorem{theorem}{Theorem}{ % frame stuff
    boxrule = 1pt,
    breakable,
    enhanced,
    frame empty,
    interior style= {orange!20},
    %interior empty,
    colframe=black,
    borderline ={1pt}{0pt}{black},
    left=0.2cm,
    % title stuff
    attach boxed title to top left={yshift=-2mm,xshift=0mm},
    coltitle=black,
    fonttitle=\bfseries,
    colbacktitle=white,
    fontupper=\slshape,
    boxed title style={boxrule=1pt,sharp corners}}{theorem}

\newtcbtheorem{corollary}{Corollary}{ % frame stuff
    boxrule = 1pt,
    breakable,
    enhanced,
    frame empty,
    interior style= {orange!20},
    %interior empty,
    colframe=black,
    borderline ={1pt}{0pt}{black},
    left=0.2cm,
    % title stuff
    attach boxed title to top left={yshift=-2mm,xshift=0mm},
    coltitle=black,
    fonttitle=\bfseries,
    colbacktitle=white,
    fontupper=\slshape,
    boxed title style={boxrule=1pt,sharp corners}}{corollary}

\newtcbtheorem{lemma}{Lemma}{ % frame stuff
    boxrule = 1pt,
    breakable,
    enhanced,
    frame empty,
    interior style= {orange!20},
    %interior empty,
    colframe=black,
    borderline ={1pt}{0pt}{black},
    left=0.2cm,
    % title stuff
    attach boxed title to top left={yshift=-2mm,xshift=0mm},
    coltitle=black,
    fonttitle=\bfseries,
    colbacktitle=white,
    fontupper=\slshape,
    boxed title style={boxrule=1pt,sharp corners}}{lemma}

\newtcbtheorem{definition}{Definition}{ % frame stuff
    boxrule = 1pt,
    breakable,
    enhanced,
    frame empty,
    interior style= {blue!10},
    %interior empty,
    colframe=black,
    borderline ={1pt}{0pt}{black},
    left=0.2cm,
    % title stuff
    attach boxed title to top left={yshift=-2mm,xshift=0mm},
    coltitle=black,
    fonttitle=\bfseries,
    colbacktitle=white,
    boxed title style={boxrule=1pt,sharp corners}}{definition}

\newtcbtheorem{exercise}{Exercise}{ % frame stuff
    boxrule = 1pt,
    breakable,
    enhanced,
    frame empty,
    interior style= {blue!6},
    %interior empty,
    colframe=black,
    borderline ={1pt}{0pt}{black},
    left=0.2cm,
    % title stuff
    attach boxed title to top left={yshift=-2mm,xshift=0mm},
    coltitle=black,
    fonttitle=\bfseries,
    colbacktitle=white,
    boxed title style={boxrule=1pt,sharp corners}}{exercise}

\newtcbtheorem{example}{Example}{ % frame stuff
    boxrule = 1pt,
    enhanced,
    frame empty,
    interior style= {green!6},%{left color=yellow!70,right color=green!70},
    %interior empty,
    colframe=black,
    borderline ={1pt}{0pt}{black},
    breakable,
    left=0.2cm,
    % title stuff
    attach boxed title to top left={yshift=-2mm,xshift=0mm},
    coltitle=black,
    fonttitle=\bfseries,
    colbacktitle=white,
    boxed title style={boxrule=1pt,sharp corners}}{example}

\theoremstyle{remark}
\newtheorem*{remark}{Remark}
\newtheorem{observation}{Observation}
%Evidenziare testo
\newtheorem*{solution}{Solution}

\newcommand\mybox[1]{%
  \fbox{\begin{minipage}{0.9\textwidth}#1\end{minipage}}}

  %Spiegazioni/verifiche
\newenvironment{greenbox}{\begin{mdframed}[hidealllines=true,backgroundcolor=green!20,innerleftmargin=3pt,innerrightmargin=3pt,innertopmargin=3pt,innerbottommargin=3pt]}{\end{mdframed}}

\newenvironment{bluebox}{\begin{mdframed}[hidealllines=true,backgroundcolor=blue!10,innerleftmargin=3pt,innerrightmargin=3pt,innertopmargin=3pt,innerbottommargin=3pt]}{\end{mdframed}}

\newenvironment{yellowbox}{\begin{mdframed}[hidealllines=true,backgroundcolor=yellow!20,innerleftmargin=3pt,innerrightmargin=3pt,innertopmargin=3pt,innerbottommargin=3pt]}{\end{mdframed}}

\newenvironment{redbox}{\begin{mdframed}[hidealllines=true,backgroundcolor=red!20,innerleftmargin=3pt,innerrightmargin=3pt,innertopmargin=3pt,innerbottommargin=3pt]}{\end{mdframed}}

\newenvironment{orangebox}{\begin{mdframed}[hidealllines=true,backgroundcolor=orange!20,innerleftmargin=3pt,innerrightmargin=3pt,innertopmargin=3pt,innerbottommargin=3pt]}{\end{mdframed}}

%emph equation
\newcommand*\myyellowbox[1]{%
  \colorbox{yellow!40}{\hspace{1em}#1\hspace{1em}}}

\newcommand*\mygreenbox[1]{%
  \colorbox{green!20}{\hspace{1em}#1\hspace{1em}}}


%%%%%%%%%%PROOF%%%%%%%%%%%%%%%%%%%%%%%%%%%%%
\usepackage{xpatch}
\xpatchcmd{\proof}{\itshape}{\normalfont\proofnamefont}{}{}
\newcommand{\proofnamefont}{\bfseries}
\renewcommand\qedsymbol{$\blacksquare$}

\newcommand\Warning{%
	\makebox[1.4em][c]{%
		\makebox[0pt][c]{\raisebox{.1em}{\small!}}%
		\makebox[0pt][c]{\color{red}\Large$\bigtriangleup$}
	}
}%





%%%%%%%%%%%%%%%%%%%%%%%%%%%%%%%%%%%%%%%%%%%%%%%%%%%%%%%%%%%%%%%%%%%%%%%%%%%%%%%%
%%%%%%%%%%%%%%%%%%%%%%%%%%%%%%%%%%%%%%%%%%%%%%%%%%%%%%%%%%%%%%%%%%%%%%%%%%%%%%%%
% SETTINGS
%%%%%%%%%%%%%%%%%%%%%%%%%%%%%%%%%%%%%%%%%%%%%%%%%%%%%%%%%%%%%%%%%%%%%%%%%%%%%%%%
%%%%%%%%%%%%%%%%%%%%%%%%%%%%%%%%%%%%%%%%%%%%%%%%%%%%%%%%%%%%%%%%%%%%%%%%%%%%%%%%
%Geometry
\makeatletter
\renewcommand{\@marginparreset}{%
    \reset@font\small
    \raggedright
    \slshape
    \@setminipage
}
\makeatother

\captionsetup[table]{font=small,labelfont={bf},skip=10pt}
\captionsetup[figure]{font=small,labelfont={bf},skip=10pt}


%link ipertestuale per indice
\hypersetup{
    colorlinks=false,
    linkcolor=black,
    filecolor=blue,
    citecolor = blue,
    urlcolor=blue,
}

%%%%%indent%%%
\setlength{\parindent}{0pt}
\setlength{\parskip}{0pt}
\setlength{\mathindent}{1.5em}

%Python in latex
\definecolor{codegreen}{rgb}{0,0.6,0}
\definecolor{codegray}{rgb}{0.5,0.5,0.5}
\definecolor{codepurple}{rgb}{0.58,0,0.82}
\definecolor{backcolour}{rgb}{0.95,0.95,0.92}
\definecolor{commentcolour}{rgb}{0.43,0.63,0.65}

\definecolor{shadecolor}{rgb}{0.93, 0.93, 0.93}
\definecolor{darkgreen}{rgb}{0.0, 0.4, 0.0}
\definecolor{darkred}{rgb}{0.8, 0.0, 0.0}
\definecolor{violet}{rgb}{0.55, 0.0, 0.55}

\definecolor{mygreen}{rgb}{0,0.6,0}
\definecolor{mygray}{rgb}{0.5,0.5,0.5}
\definecolor{mymauve}{rgb}{0.58,0,0.82}
\lstset{
    backgroundcolor=\color{shadecolor},       % background color
    basicstyle=\ttfamily\footnotesize,        % the size of the fonts that are used for the code
    breakatwhitespace=false,                  % sets if automatic breaks should only happen at whitespace
    breaklines=false,                         % sets automatic line breaking
    captionpos=b,                             % sets the caption-position to bottom
    commentstyle=\color{mygreen},             % comment style
    extendedchars=true,                       % lets you use non-ASCII characters; for 8-bits encodings only, does not work with UTF-8
    keepspaces=true,                          % keeps spaces in text, useful for keeping indentation of code (possibly needs columns=flexible)
    keywordstyle=\bfseries\color{blue},       % keyword style
    language=[95]Fortran,                     % the language of the code
    numbers=left,                             % where to put the line-numbers; possible values are (none, left, right)
    numbersep=5pt,                            % how far the line-numbers are from the code
    numberstyle=\tiny\color{mygray},          % the style that is used for the line-numbers
    rulecolor=\color{black},                  % if not set, the frame-color may be changed on line-breaks within not-black text (e.g. comments (green here))
    showspaces=false,                         % show spaces everywhere adding particular underscores; it overrides 'showstringspaces'
    showstringspaces=false,                   % underline spaces within strings only
    showtabs=false,                           % show tabs within strings adding particular underscores
    stepnumber=1,                             % the step between two line-numbers. If it's 1, each line will be numbered
    stringstyle=\color{mymauve},              % string literal style
    tabsize=4,                                % sets default tabsize to 2 spaces
    title=\lstname                            % show the filename of files
}
