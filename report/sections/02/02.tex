\providecommand{\main}{../../main}
\documentclass[../../main/main.tex]{subfiles}


\begin{document}

\section{Experimental apparatus}
The experimental apparatus employed for the experience composed of:
\begin{itemize}
    \item a vacuum system, including a vacuum chamber, a turbomolecular pump to reach the vacuum condition and two vacuumeters to keep track of the internal pressure of the chamber;
    \item a mechanical system, including a step motor through which the support of the source is rotated. This is connected to an Arduino shield and to a computer, through which the step position can be regulated;
    \item a radioactive source of Am-241, with active material inside an aluminium cylinder;
    \item a partially depleted silicon surface barrier detector, connected to a NIM module electronic chain and then to a Picoscope digitizer to sample the waveform of the candidate \( \alpha \) signals;
    \item an ALPIDE detector.
\end{itemize}



\subsection{Vacuum system}
The core component of the vacuum system, in which the scattering process takes place, is the vacuum chamber. It has a cylindrical form and its internal diameter is of about \( 22 \ \si{cm} \). It is connected to the double phase turbomolecular pump Pfeiffer Vacuum.


\subsection{Mechanical system}

\subsubsection{Step motor}

\subsubsection{Source support}






\subsection{\boldmath \( \alpha  \) particles source}



\subsection{Silicon detector}



\subsection{ALPIDE detector}

\end{document}
